\newpage
\begin{center}
  \textbf{\large 1. ТЕОРИТИЧЕСКАЯ ЧАСТЬ}
\end{center}
\refstepcounter{chapter}
\addcontentsline{toc}{chapter}{1. ТЕОРИТИЧЕСКАЯ ЧАСТЬ}


\section{Литературный обзор}
Тема составления расписания активно изучается. Создаются различные алгоритмы и программное обеспечение для составления расписания в образовательных учреждениях.

В одной из работ приведена математическая модель распределения академической нагрузки (ALD), в которой учебные курсы состоят из отдельных блоков  \cite{zaozerskaya2018modeling}. Каждый из блоков может быть назначен только одному преподавателю. Минимально и максимально возможные объемы академической нагрузки устанавливаются для каждого преподавателя. Предложены модели целочисленного линейного программирования (ILP) для различных формулировок этой задачи. Одна из моделей учитывает межличностные отношения при назначении преподавателей по одной дисциплине

В статье \cite{delaRosaRivera2020} предлагается набор показателей для прогнозирования производительности алгоритмов комбинаторной оптимизации, которые генерируют исходные решения для экземпляров университетского расписания. Приведены следующие условия: каждому классу должен быть назначен учитель и аудитория, ресурсы (например, учебные планы, преподаватели и аудитории) не должны быть распределены по разным лекциям одновременно и т.д.

Задача о распределении работников с несколькими квалификациями в контексте производственных систем seru, в которых учитываются различия в наборах навыков и уровнях квалификации работников рассмотрена в научной работе \cite{Lian2018}. Распределение работников по группам, загрузка ячеек и назначение задач решаются в задаче одновременно. Авторами предложена математическая модель улучшения баланса рабочей нагрузки между сотрудниками.

В \cite{zakharova2022integer} рассматривается задача составления расписания профориентационной школы и исследуются ее особенности. Описываются факторы со стороны образовательного учреждения, детей и родителей, влияющие на формирование групп и временные диапазоны для проведения занятий. Приводится математическая модель составления групп и расписания в школах дополнительного образования.


\section{Анализ существующих решений}

Рассмотрим существующие программные продукты позволяющие формировать автоматическое расписание.

\begin{table}[h!]
\centering
\begin{tabular}{|p{3.5cm}|p{2.5cm}|p{2.5cm}|p{2.5cm}|}
\hline
\textbf{Название} & \textbf{Цена} & \textbf{Группы} & \textbf{Расписание} \\
\hline
Электронные таблицы в пакетах OpenOffice,
LibreOffice и др. & бесплатно & нет & нет \\
\hline
1С:Университет & от 95 тыс руб & нет & да \\
\hline
Мой Класс & от 690 руб/мес & нет & да \\
\hline
EduTerra.PRO & от 1199 руб/мес & нет & да \\
\hline
\end{tabular}
\caption{Сравнение продуктов}
\label{table:products}
\end{table}

Такие программы, как 1С:Университет, GS-Ведомости,Мой Класс и EduTerra PRO предлагают решения для составления расписания, но так как они не направлены на сферу дополнительного образования, они не учитывают особенности составления расписания, пожелания учеников и родителей, в них не предусмотрен функционал составления групп обучающихся с учетом пожеланий как преподавателей, так и родителей учеников. Большинство существующих программ предназначены для учебных заведений общего и высшего образования, их функционал не подходит для школ дополнительного образования, либо требует дополнительной модификации, что ведет к повышению стоимости как самого продукта, так и его обслуживания. Поэтому разработка приложений для организации учебного процесса дополнительного образования является актуальной задачей. В данной работе исследуется алгоритм решения задачи составления расписания, который может быть использован в подобных приложениях.


\section{Цели и задачи бакалаврской работы}

\textbf{Цель работы} -- 

\textbf{Задачи работы:}
\begin{enumerate}
\item 
\item 
\item 
\item 
\end{enumerate}
